\documentclass{aip-cp}

\usepackage[numbers]{natbib}
\usepackage{rotating}
\usepackage{graphicx}
\usepackage{caption}

% Document starts
\begin{document}

% Title portion
\title{Light Meson Decays from Photon-Induced Reactions with CLAS}

\author[aff1]{Michael C. Kunkel\corref{cor1}}
\author[]{for the CLAS Collaboration }
\affil[aff1]{Forschungszentrum J\"ulich, J\"ulich (Germany)}
\corresp[cor1]{m.kunkel@fz-juelich.de}

\maketitle

\begin{abstract}
Photo-production experiments with the CEBAF Large Acceptance Spectrometer (\textsc{\texttt{CLAS}}) at the Thomas Jefferson National Laboratory produce data sets with unprecedented statistics of light mesons. With these data sets, measurements of transition form factors for $\eta$, $\omega$, and $\eta^\prime$ via conversion decays can be performed using a line shape analysis on the invariant mass of the final state dileptons. Tests of fundamental symmetries and information on the light quark mass difference can be performed using a Dalitz plot analysis of the meson decay. An overview of the first results and future prospects within the newly upgraded \textsc{\texttt{CLAS}} apparatus will be given.
\end{abstract}

% Head 1
\section{INTRODUCTION}
Decays of light mesons provide insight into the structure of the meson. The Light Meson Decay (LMD) group, established at the Thomas Jefferson National Facility with worldwide collaboration, investigates physics pertaining to, but not limited to, transition form factors, anomalous decays and the search for CP violation through Dalitz plot analysis. The presentation given was an overview of the LMD program, recent updates on measurements and an outlook on measurements that can be taken with the CLAS12 detector. 

\section{Light Meson Decay Program}
The light meson group was established in 2013. The goal of the group is to investigate properties of light meson decays using data obtained from the CLAS detector. Figure~\ref{fig:clas} shows the CLAS detector and its sub components.
\begin{figure}[h]
	\centerline{\includegraphics[width=175 pt]{figures/clas_schematicIII.pdf}}
	\caption{The CEBAF Large Acceptance Spectrometer (CLAS) }
	\label{fig:clas}
\end{figure}
 Since decays of hadrons are independent of production, all CLAS data can be used, however there are two experiments that were chosen as flagships for the program, the g11 and g12 experiment. Both experiments use a photon beam incident on a liquid hydrogen target which created photo-induced reactions, 800~MeV - 3.8~GeV for g11 and 1.1~GeV - 5.5~GeV for g12.  See~\cite{lmdCAA} for a complete list of meson decays the LMD group plans to investigate.
% See Table~\ref{tab:lmd.channels} for a list of meson decays the LMD group plans to investigate.
%\begin{table}[h!]
\begin{minipage}{\textwidth}
\begin{center}


\caption{LMD planned measurements \vspace{0.75mm}}

\begin{tabular}{cc||cc}
%\begin{tabular}{p{5cm} | p{7cm}}
\hline
Meson Decay & Physics Interest &Meson Decay & Physics Interest \\
\hline
$\pi^0\to e^+e^-\gamma$  & Heavy photon upper limit &$\eta^{\prime}\to \pi^+\pi^-\gamma$  & Box anomaly \\
$\eta^{\prime}\to e^+e^-\gamma$  & Transition form factor &$\omega\to \pi^+\pi^-\gamma$  & Upper limit branching ratio \\
$\omega\to e^+e^-\pi^0$ & Transition form factor & $\eta, \omega, \phi\to \pi^+\pi^-\pi^0$ & Dalitz plot analysis\\
$\eta^{\prime}\to e^+e^-\pi^0$ & C violation & $\eta^{\prime}\to \pi^+\pi^-\eta0$ & Dalitz plot analysis\\
$\eta^{\prime}\to e^+e^-\pi^+\pi^-$  & CP violation & $\phi\to \pi^+\pi^-\eta0$ & G-parity violation\\
\hline 
\end{tabular}


\end{center}
\end{minipage}
\end{table}
\vspace{20pt}
\subsection{Update on the Radiative decay of the $\eta$ and $\eta^\prime$  meson}
The 2 photon decay of pseudoscalar mesons $\pi^0, \eta , \eta^{\prime} \to \gamma \gamma $ proceed from the understood triangle or axial anomaly. While radiative decays of  $\eta , \eta^{\prime} \to \pi^+ \pi^- \gamma $ are related to a less understood box anomaly.
%\begin{figure}[h]
%	\begin{minipage}{.35\textwidth}
%		\centering
%		\centerline{\includegraphics[width=75 pt]{figures/triangleIII.pdf}}
%		\caption{}{A. Triangle diagram}
%		\label{fig:test1}
%	\end{minipage}%
%	\begin{minipage}{.35\textwidth}
%		\centering
%		\centerline{\includegraphics[width=75 pt, height=50 pt]{figures/boxIII.pdf}}
%		%\caption{figure in here}{box diagram}
%		\caption{Feynmann diagram of the two photon decay (A). Feynmann diagram of the axial anomoly, box diamgram (B)}{B. Box diagram}
%		\label{fig:decays}
%	\end{minipage}
%\end{figure}
The  radiative decay widths of $ \eta^{\prime}$ and $\eta^{\prime}$ are determined by the box anomaly in the chiral limit by use of equation~\ref{eq:decaywidth}.

%An analysis of the photon energy distribution of the radiative decays of $ \eta^{\prime}$ and $\eta^{\prime}$, the decay widths are determined by the box anomaly in the chiral limit.

\begin{equation}

\frac{d\Gamma (\eta^{(\prime)} \to \pi^+ \pi^- \gamma)}{ds_{\pi\pi}} = A\vert P(s_{\pi\pi}) F_V(s_{\pi\pi}) \Gamma_0(s_{\pi\pi})\vert  \label{eq:decaywidth} \\

\end{equation}

Where $\Gamma_0(s_{\pi\pi})$ is the P-wave phase-space constant, denoted in equation~\ref{eq:decayconstant} with $\kappa$ being a numerical constant. $F_V(s_{\pi\pi})$ is the pion form factor that can be approximated by the equation~\ref{eq:decayformfactor} and  $P(s_{\pi\pi})$ is expanded in the chiral limit, $s_{\pi\pi} = 0$, and is written in equation~\ref{eq:decaychiral}, where $\alpha$ is the variable of measurement.
\begin{eqnarray}
\Gamma_0(s_{\pi\pi}) = \frac{\kappa \left(M^2_{\eta^{(\prime)}} - s_{\pi\pi} \right)^3  s_{\pi\pi} \left(1- \frac{ 4M^2_{\pi }}{    s_{\pi\pi}  }\right)^{\frac{3}{2}}   }{M^3_{\eta^{(\prime)} }}  \label{eq:decayconstant}  \\
\vert F_V(s_{\pi\pi}) \vert \approx 1+(2.12\pm0.01)s_{\pi\pi} + (2.13\pm0.01)s_{\pi\pi}^2+(13.89\pm0.14)s_{\pi\pi}^3 \label{eq:decayformfactor} \\
P(s_{\pi\pi}) = 1 + \alpha s_{\pi\pi} + \mathcal{O}(s_{\pi\pi}^2) \label{eq:decaychiral}
\end{eqnarray}
Previous measurements of the radiative decay for the $\eta$ meson from WASA-at-COSY~\cite{bib0} and KLOE~\cite{bib1} differ in such a manner that a third measurement is needed, furthermore there exists only one measurement, performed  of the $\eta^{\prime}$ radiative decay~\cite{bib2}. With the CLAS g11 experiment, both the $\eta$ and  $\eta^{\prime}$ radiative decay width will be measured. In figure~\ref{fig:boxCLASdata} the CLAS g11 data is shown for the particle selection of exclusive $\gamma p \to p  \pi^+ \pi^- \gamma $. Selecting events withing a $2.5 \sigma$ range of $\eta^{(\prime)}$ the photon energy distribution is shown in figure~\ref{fig:boxCLAS}. 
\begin{figure}[h]
	\centerline{\includegraphics[width=175 pt]{figures/clas_g11data.pdf}}
	\caption{CLAS data yield for $\gamma p \to p \eta^{(\prime)} \to p \pi^+ \pi^- \gamma $ from g11 data set }
	\label{fig:boxCLASdata}
\end{figure}
\begin{figure}[h]
	\centerline{\includegraphics[width=275 pt]{figures/Box_CLAS.pdf}}
	\caption{CLAS data photon energy distribution for $\eta$ (left) and $\eta^{\prime}$ (right)}
	\label{fig:boxCLAS}
\end{figure}
Visually comparing the shape of the left figure~\ref{fig:boxCLAS} to those of figure~\ref{fig:kloe_eta}, for the  $\eta$ meson, and also the right figure~\ref{fig:boxCLAS} to that of figure~\ref{fig:crystal_etaP}, for the $\eta^{\prime}$ meson, it can be seen that the CLAS data is suitable for comparison with previous measurements.
\begin{figure}[h!]
	\centering
	\begin{minipage}{.30\textwidth}
		\centering
		\includegraphics[width=125 pt]{figures/WASA_eta.pdf}
		\caption{}{}
		\label{fig:wasa_eta}
	\end{minipage}%
	\centering
	\begin{minipage}{.30\textwidth}
		\centering
		\includegraphics[width=125 pt, height = 100 pt]{figures/KLOE_eta.pdf}
		%\caption{figure in here}{box diagram}
		\caption{WASA-at-COSY data photon energy distribution for $\eta$ (left)~\cite{bib3} and KLOE data photon energy distribution for $\eta$ (right)\cite{bib2}.}{}
		\label{fig:kloe_eta}
	\end{minipage}
\end{figure}
\begin{figure}[h!]
	\centerline{\includegraphics[width=135 pt]{figures/CRYSTAL_etaP.pdf}}
	\caption{CRYSTAL BARREL photon energy distribution for  $\eta^{\prime}$~\cite{bib3}}
	\label{fig:crystal_etaP}
\end{figure}
%\newpage
\subsection{Update on the Dalitz plot analysis of $\eta^{\prime} \to \pi^+ \pi^- \eta$}
%The Dalitz plot of $\eta^{\prime} \to \pi^+ \pi^- \eta$ provides kinematic information of the decay, enabling the study of low energy dynamics of QCD and heavier mass pseudoscalar mesons. The  $\eta^{\prime} \to \pi^+ \pi^- \eta$ decay has a low Q-value due to the decay products being relatively heavy, this helps test and limit the effective chiral Lagrangian theory. The measurable of the Dalitz plot $X$ and $Y$ are projected and fitted to the equation~\ref{eq:dalitzpro} to extract the parameters $a$, $b$, $c$, $d$. Table~\ref{tab:dal.parms} shows the previous measurements of equation~\ref{eq:dalitzpro} along with the projected statistical error a measurement from the CLAS g11 data.
%
\begin{equation}
%
f(X,Y) = A(1+a(Y) + b (Y^2) + c(X) + d(X^2)  \label{eq:dalitzpro} \\
%
\end{equation}
%
\begin{table}[h!]
\begin{minipage}{\textwidth}
\begin{center}


\caption{\label{tab:dal.parms}Previous \vspace{0.75mm}}
\begin{tabular}{cccccc}
%\begin{tabular}{p{5cm} | p{7cm}}
\hline
Parameter & VES & Theory & BESIII & Stat. err. in BESIII & Projected CLAS stat. err. \\
\hline
a & -0.127 $\pm$ 0.018 & -0.116 $\pm$ 0.011 & -0.047 $\pm$ 0.012 &$\pm$ 0.011&$\pm$ 0.004 \\
b & -0.106 $\pm$ 0.032 & -0.042 $\pm$ 0.034 & -0.069 $\pm$ 0.021 &$\pm$ 0.019&$\pm$ 0.006\\
c & 0.015                        &                                  & 0.019  $\pm$  0.012 &$\pm$ 0.011&$\pm$ 0.004\\
d & -0.082 $\pm$ 0.019 & -0.010 $\pm$ 0.019 & -0.073 $\pm$ 0.013 &$\pm$ 0.012&$\pm$ 0.004\\
\end{tabular}


\end{center}
\end{minipage}
\end{table}
\vspace{20pt}
This topic was briefly discussed, as a full update was given by later in the session.
\subsection{Update on the transistion form factor measurement  of the $\omega$ meson}
Transition form factors characterize modifications of the point-like photon-meson vertex due to the structure of the meson. Since the virtual photon can interact with quarks, it can be used as a probe for the internal structure of mesons and its electromagnetic interaction is calculable with the Kroll-Wada formula~\cite{bib4} as seen in equation~\ref{eq:kroll};
\begin{equation}
\frac{d\Gamma_{M{\rightarrow l^{+}l^{-}X}}}{dq^{2} d\Gamma_{M{\rightarrow X\gamma}}} = \frac{\alpha}{3\pi q^{2}}\left(\left(1+\frac{q^{2}}{m^{2}_{M}-m^{2}_{X}}\right)^2 - \frac{4m^{2}_{M}q^2}{(m^{2}_{M}-m^{2}_{X})^2}\right)^\frac{3}{2}\left(1-\frac{4m_{l}^{2}}{q^{2}}\right)^{1/2}\left(1+\frac{2m_{l}^{2}}{q^{2}}\right) \vert_{\mathrm{Q.E.D}}  \label{eq:kroll} \ . \\
\end{equation}
 Where $M$ is the species of meson i.e. $\pi^0$, $\eta$, $\omega$, $\eta^{\prime}$, etc, $X$ is the daughter particle in the decay, $m_M$ the mass of the meson, $m_X$ the mass of the daughter particle, $m_l$ the mass of the lepton species in the decay, i.e. $e^{\pm}$ or $\mu^{\pm}$ and $q$ being the $e^{+}e^{-}$ or $\mu^{+}\mu^{-}$ invariant mass which is equivalent to the mass of the virtual photon. Deviation of equation~\ref{eq:kroll} represents the  architecture of the meson and can be incorporated into the transition form factor $\left| F(q^2)\right|$. Depending on the lifetime of the meson of interest, the transition can be constructed as a simple pole, equation~\ref{eq:pole}, a complex pole, equation~\ref{eq:cpole}, or some other arrangement that describes the transition.
 \begin{eqnarray}
\left| F(q^2)\right| = \frac{1}{1-\frac{q^2}{\Lambda^2}} \label{eq:pole} \\
\left| F(q^2)\right|^2 = \frac{\Lambda^2(\Lambda^2 + \gamma^2)}{(\Lambda^2 - q^2)\Lambda^2 \gamma^2} \label{eq:cpole} \ ,
 \end{eqnarray} 
 where $\Lambda$ and $\gamma$ is the mass and width of the virtual vector meson mass,respectively.
 Recent measurements of the transition form factor for $\omega \to \mu^+\mu^- \gamma$ have shown discrepancy with the Vector Dominance Model~\cite{bib5} and recent models~\cite{bib6} attempt to predict the shape of the virtual vector meson as seen in figure~\ref{fig:omega_ff}.
 \begin{figure}[h!]
 	\centerline{\includegraphics[width=140 pt, height=110 pt]{figures/omega_ff.pdf}}
 	\caption{Form factor of the decay $\omega \to l^+l^- \gamma$ compared to dimuon data taken by the NA60 Collaboration ~\cite{bib5}. The solid line and dotted line represent recent theorteical models to describe the line shape. The dot-dashed line is calculated with the VMD model equations~\ref{eq:kroll},~\ref{eq:cpole} using the mass of the $\rho$ meson as the virtual vector meson.~\cite{bib6}}
 	\label{fig:omega_ff}
 \end{figure}
With the CLAS g12 experiment, lepton ($e^{\pm}$) identification is done using Cherenkov detectors and electromagnetic calorimetry, providing a $e^{+}e^{-}/\pi^{+}\pi^{-}$ rejection of $10^6$. Using the $e^{\pm}$ data from CLAS g12, the $ \omega \to p e^+ e^- \pi^0$ transition form factor can be extracted to validate onto the discrepancy. Figure~\ref{fig:clas_omega_ff_q} shows the mass section for the $\omega$ meson as well as the $e^+ e^-$ distribution currently being used to measure the $\omega$ transition form factor.  
\begin{figure}[h!]
	\begin{minipage}{.35\textwidth}
		\centering
		\centerline{\includegraphics[width=125 pt]{figures/clas_omega_ff.pdf}}
		\caption{}{}
		\label{fig:clas_omega_ff}
	\end{minipage}%
	\begin{minipage}{.35\textwidth}
		\centering
		\centerline{\includegraphics[width=115 pt]{figures/clas_omega_ff_q.pdf}}
		%\caption{figure in here}{box diagram}
		\caption{CLAS data yield for $\gamma p \to p X \to p e^+ e^- \pi^0 $ from g12 data set (left). $q=e^+ e^-$ distribution for events  $\gamma p \to p \omega \to p e^+ e^- \pi^0 $ from the g12 data set (right).}{}
		\label{fig:clas_omega_ff_q}
	\end{minipage}
\end{figure}
Also, the knowledge of the $\eta$ form factor is needed for the interpretation of the g-2 experiment. Furthermore, the ratio of the $\eta$ and $\eta^{\prime}$ transition form factor provides information into the mixing angle of the combination of the singlet, $\eta_0$, and nonet, $\eta_8$, which are the composites of the $\eta$ and $\eta^{\prime}$ meson.
%\newpage
\subsection{Update on the branching ratio measurement  of the $\eta^\prime$ meson $\rightarrow e^+e^-\gamma$}
A recent measurement of BESIII reports a branching ratio $\Gamma(\eta^{\prime} \to  e^+ e^-  \gamma)$/$\Gamma(\eta^{\prime} \to  \gamma  \gamma)$ to be $2.13\pm0.09(stat.)\pm0.07(sys.))10^{-2}$ from 864 events~\cite{bib7}.  Using the $e^{\pm}$ data from CLAS g12, preliminarily 89 events of the $\eta^{\prime} \to  e^+ e^-  \gamma$ decay were observed and analyzed, using the Q-factor~\cite{bib8} method to suppress background from neighboring $A \to e^+ e^-  X$ decays, figure~\ref{fig:etaP_ff}.
 \begin{figure}[h!]
 	\centerline{\includegraphics[width=275 pt, height=125 pt]{figures/clas_etaP_ff.pdf}}
 	\caption{CLAS data yield for $\gamma p \to p \eta^{\prime}  \to p e^+ e^- \gamma $ from g12 data set. Blue peak represtns the signal extracted using the Q-factor method, while the gray band represents the background (left). $q=e^+ e^-$ distribution for events  $\gamma p \to p \eta^{\prime}  \to p e^+ e^- \gamma $ from the g12 data set. Black points represent data under the blue peak of left plot, while the blue-dashed is simulation of $\eta^{\prime} \to  \gamma  \gamma$ and the red solid area is the simulation of $\eta^{\prime} \to  e^+ e^-  \gamma$(right).}
 	\label{fig:etaP_ff}
 \end{figure}
  A preliminary branching ratio $\Gamma(\eta^{\prime} \to  e^+ e^-  \gamma)$/$\Gamma(\eta^{\prime} \to  \gamma  \gamma)$ was measured to be $2.63\pm0.3(stat.))10^{-2}$, which is consistent with the BESIII measurement. However, the statistics of either BESIII or CLAS are not enough to determine which theoretical model represents the structure of the $\eta^{\prime}$ meson~\cite{bib10,bib11,bib12}, table~\ref{tab:etaP.models} outlines the current measurements and theoretical predictions.
\begin{table}[h!]
\centering
\caption{\label{tab:etaP.models}Current Measurements and Theoretical Predictions of the $\eta^{\prime}$ charge radius \vspace{0.75mm}}

\begin{tabular}{cc}
%\begin{tabular}{p{5cm} | p{7cm}}
\hline
Charge Radius $\left< r\right>$ & Measurement (M) / Prediction (P) \\
\hline
CLAS ($\eta^{\prime}\to e^+e^-\gamma$)   & TBD \\
BESIII ($\eta^{\prime}\to e^+e^-\gamma$)   & (M) $1.60\pm0.17(stat)\pm0.08(sys) \mathrm{GeV}^{-2}$~\cite{bib7} \\
CELLO ($\eta^{\prime}\to \mu^+\mu^-\gamma$)   & (M)  $1.7\pm0.4 \mathrm{GeV}^{-2}$~\cite{bib9}  \\
\hline
Dispersion    & (P)  $1.53^{0.15}_{-0.08} \mathrm{GeV}^{-2}$~\cite{bib12}  \\
ChPT    & (P) $1.6 \mathrm{GeV}^{-2}$~\cite{bib11}  \\
VMD    & (P) $1.45  \mathrm{GeV}^{-2}$~\cite{bib10}  \\
\hline 
\end{tabular}
\end{table}
%\vspace{20pt}
%
\subsection{Future measurement of the $\eta^\prime$ meson transsition form factor with CLAS12}
With the newly built CLAS12 detector, $e^{\pm}$ identification can be achieved with a $e^{+}e^{-}/\pi^{+}\pi^{-}$ rejection of  $10^{6}-10^{11}$ while retaining $e^+ e^- \gamma$ acceptance $\sim 1 \% - 0.1 \%$, figure~\ref{fig:clas12} depicts the CLAS12 detector and its subsystems.
\begin{figure}[h!]
	\centerline{\includegraphics[width=200 pt, angle = 90]{figures/clas12-design.pdf}}
	\caption{The CEBAF Large Acceptance Spectrometer (CLAS12)\\~(https://www.jlab.org/Hall-B/clas12-web/)}
	\label{fig:clas12}
\end{figure}
Using the GEant4 Monte-Carlo (GEMC) simulation suite for CLAS12, a simulation of 600,000 $e p \to e^{\prime} \gamma^* p \to p \eta^{\prime}  \to p e^+ e^- \gamma $ was performed. The generation of events included cross-section information obtained from previous CLAS measurements and the $s^n$ scaling law on the cross-section to achieve a modest model of the production of the $\eta^{\prime} $ meson. Also in the generation the VMD model was incorporated for the decay of the $\eta^{\prime}$ meson. The estimated quasi-real photon rate with the CLAS12 Forward Tagger are $5 \cdot 10^7 \gamma / s$ which will be impinged on  a 5~cm $\ell H_2$ target. Count rates of $ \eta^{\prime}  \to p e^+ e^- \gamma $ were calculated for exclusive $\gamma p \to p \eta^{\prime}  \to p e^+ e^- \gamma $ and inclusive $\gamma p \to \eta^{\prime} (p)  \to  e^+ e^- \gamma (p) $ with and without the electromagnetic calorimeter (EC) information for the $ e^+ e^- $ pairs. For the exclusive reaction it was preliminarily estimated that the number of  $ \eta^{\prime}  \to p e^+ e^- \gamma $ decays to be detected within 100 days of beam time would be $\sim 22,000 - 2,400$, and the statistical uncertainty of a the transition form factor measurement would be $\sim 0.3 \% - 3 \%$, with and without the EC information respectively, figure~\ref{fig:clas12_etaP}. For the inclusive reaction it was preliminarily estimated that the number of  $ \eta^{\prime}  \to p e^+ e^- \gamma $ decays to be detected within 100 days of beam time would be $\sim 53,000 - 5,900$, and the statistical uncertainty of a the transition form factor measurement would be $\sim 0.1 \% - 1 \%$, with and without the EC information respectively, figure~\ref{fig:clas12_etaP}. Furthermore, with the statistics estimated to be seen in CLAS12, a $ \eta^{\prime} $ signal is expected in the complete range of $q = e^+ e^- $, allowing for a line shape measurement at the divergent part of the VMD model $q\sim m_{\rho}$. This will give greater insight into the structure of the $ \eta^{\prime} $ meson than previously measured. 
\begin{figure}[h!]
	\centerline{\includegraphics[width=300 pt, height=200 pt]{figures/clas12_etaP.pdf}}
	\caption{Expected count rates for $ \eta^{\prime}  \to p e^+ e^- \gamma $ in CLAS12}
	\label{fig:clas12_etaP}
\end{figure}
%\begin{figure}[h]
%	\centerline{\includegraphics[width=200 pt, height= 100 pt]{figures/clas12_etaP_recon.pdf}}
%	\caption{The CEBAF Large Acceptance Spectrometer (CLAS12)}
%	\label{fig:clas12_etaPrecon}
%\end{figure}
% Acknowledgement
\section{ACKNOWLEDGMENTS}
I would like to acknowledge the members of the LMD group for their contributions to the given presentation. Also to the CLAS collaboration.

% References

\nocite{*}
\bibliographystyle{aipnum-cp}%
\bibliography{LMD}%


\end{document}
