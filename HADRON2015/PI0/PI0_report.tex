\documentclass{aip-cp}

\usepackage[numbers]{natbib}
\usepackage{rotating}
\usepackage{graphicx}
\usepackage{caption}

% Document starts
\begin{document}

% Title portion
\title{Light Meson Decays from Photon-Induced Reactions with CLAS}

\author[aff1]{Michael C. Kunkel\corref{cor1}}
\author[]{for the CLAS Collaboration }
\affil[aff1]{Forschungszentrum J\"ulich, J\"ulich (Germany)}
\corresp[cor1]{m.kunkel@fz-juelich.de}

\maketitle

\begin{abstract}
Photoproduction of the $\pi^0$ meson was studied using the \textsc{\texttt{CLAS}} detector at Thomas Jefferson National Accelerator Facility using tagged incident beam energies spanning the range $E_{\gamma}=$~1.1~GeV~-~5.45~GeV. The measurement is performed on a liquid hydrogen target in the reaction $\gamma p\to pe^+e^-(\gamma)$. The final state of the reaction is the sum of two subprocesses for $\pi^0$ decay, the Dalitz decay mode of $\pi^0\to e^+e^-\gamma$ and conversion mode where one photon from $\pi^0\to \gamma\gamma$ decay is converted into a $e^+e^-$ pair. This specific final state reaction avoided limitations caused by single prompt track triggering and allowed a kinematic range extension to the world data on $\pi^0$ photoproduction to a domain never systematically measured before.

We report the measurement of the $\pi^0$ differential cross-sections $\frac{d\sigma}{d\Omega}$ and $\frac{d\sigma}{dt}$. The angular distributions agree well with the SAID parametrization for incident beam energies below 3~GeV, while an interpretation of the data for incident beam energies greater than 3~GeV is currently being developed. Included in the report will be a discussion of the future wide angle, exclusive photoproduction of $\pi^0$ experiment that will be performed in Hall C.
\end{abstract}

% Head 1
\section{INTRODUCTION}
In hadron physics, photoproduction of single pion is essential to understand the photon-nucleon vertex. At low energies, the photon-nucleon coupling establishes excited nucleon resonances which has been at the forefront of physics ''missing resonances'' search. At high energies single pion photoproduction can be used to test predictions of Regge theory, in which recent calculations~\cite{bib0} have shown to describe the presented data well. The following proceeding detailed the CLAS g12 experiment, the extraction of single neutral pion photoproduction from data, the differential cross-sections through the entire beam energy range of the g12 experiment, and finally a comparison of the differential cross-section with existing world data, as well as the comparison of the data to the model given in~\cite{bib0}.
\section{CLAS}
\section{The g12 Experiment}
\subsection{Particle Selection}
\subsection{Kinematic Cuts}
\subsection{Measurements}
\section{Comparison to Existing Data}
\section{Comparison to Regge Model}
\section{$s^7$ Scaling}
\subsection{HALL C at Jlab12 Outlook}
% Acknowledgement
\section{ACKNOWLEDGMENTS}
I would like to acknowledge the members of the LMD group for their contributions to the given presentation. Also to the CLAS collaboration.

% References

\nocite{*}
\bibliographystyle{aipnum-cp}%
\bibliography{PI0}%


\end{document}
